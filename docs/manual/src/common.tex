\begin{titlepage}
  \begin{center}

  {\Huge Mini Colecovision}

  \vspace{25mm}

  \includegraphics[width=0.90\textwidth,height=\textheight,keepaspectratio]{img/SPARKLETRON.png}

  \vspace{25mm}

  \today

  \vspace{15mm}

  {\Large Jay Convertino}

  \end{center}
\end{titlepage}

\tableofcontents

\newpage

\section{Introduction}

\par
Mini Colecovision is a portable console version of the original Colecovision. Much of the TTL and Analog Monostable circuits
are emulated by a CPLD. The full PCB and CPLD code is in this repository. It emulates the original Colecovision
with the additional super game module. No 3D printed case is included at the moment, but has been designed and tested. This
manual is not at step by step document of how to build the unit, more of a highlight of aspects of the project.

\subsection{Specifications}

\par
\begin{itemize}
  \item Z80 CPU
  \item 32 KiB of RAM
  \item 32 KiB of ROM
  \item SN76489 Sound Chip
  \item YMZ284 Sound Chip
  \item TMS9118 Video Display Processor with 16 KiB of VRAM
  \item MAX7000S CPLD (EPM7128SLC)
  \item Main PCB, four layer
  \item Right Angle PCB, two layer
\end{itemize}

\subsection{Parts List}

\begin{footnotesize}
\begin{longtable}{ |*{4}{c|} }
\hline
{Item} & {Qty} & {Reference(s)} & {Value} \\
\hline
{1} & {13} & {C1, C7, C8, C10 to C14, C18, C19, C23, C25, C29} & {100nF} \\
\hline
{2} & {1} & {C2} & {330uF} \\
\hline
{3} & {4} & {C3, C5, C24, C30} & {10uF} \\
\hline
{4} & {1} & {C4} & {100pF} \\
\hline
{5} & {1} & {C6} & {270pF} \\
\hline
{6} & {8} & {C9, C17, C22, C31 to C35} & {100nF} \\
\hline
{7} & {2} & {C15, C16} & {33pF} \\
\hline
{8} & {2} & {C20, C21} & {10nF} \\
\hline
{9} & {1} & {C26} & {0.47uF} \\
\hline
{10} & {1} & {C27} & {0.1uF} \\
\hline
{11} & {1} & {C28} & {2.2uF} \\
\hline
{12} & {1} & {D1} & {LED} \\
\hline
{14} & {1} & {J1} & {Conn\_01x07} \\
\hline
{15} & {3} & {J2, J10, J11} & {Conn\_01x02} \\
\hline
{16} & {1} & {J3} & {Conn\_Coaxial} \\
\hline
{17} & {1} & {J4} & {Cartridge Port} \\
\hline
{18} & {1} & {J5} & {DB9 Male} \\
\hline
{19} & {1} & {J6} & {Conn\_02x05\_Odd\_Even} \\
\hline
{20} & {1} & {J7} & {DB9 Male} \\
\hline
{21} & {1} & {J9} & {SJ1-3525NG} \\
\hline
{22} & {3} & {L1, L2, L3} & {4.7uH} \\
\hline
{23} & {1} & {Q1} & {2N3904} \\
\hline
{24} & {1} & {R1} & {4k7} \\
\hline
{25} & {1} & {R2} & {470R} \\
\hline
{26} & {2} & {R3, R28} & {100k} \\
\hline
{27} & {2} & {R4, R27} & {100K} \\
\hline
{28} & {2} & {R5, R6} & {2k2} \\
\hline
{29} & {3} & {R7, R8, R9} & {3K3} \\
\hline
{30} & {1} & {R10} & {75R} \\
\hline
{31} & {1} & {R11} & {510R} \\
\hline
{32} & {1} & {R12} & {100R} \\
\hline
{33} & {1} & {R13} & {3k3} \\
\hline
{34} & {4} & {R14, R15, R16, R17} & {1k} \\
\hline
{35} & {2} & {R18, R19} & {10k} \\
\hline
{36} & {1} & {R20} & {1K} \\
\hline
{37} & {1} & {R21} & {1k} \\
\hline
{38} & {1} & {R22} & {220R} \\
\hline
{39} & {1} & {R23} & {10K} \\
\hline
{40} & {2} & {R24, R26} & {1K} \\
\hline
{41} & {1} & {R25} & {68K} \\
\hline
{42} & {2} & {RN1, RN2} & {10k} \\
\hline
{43} & {1} & {RV1} & {10K} \\
\hline
{45} & {1} & {SW1} & {SW\_Push} \\
\hline
{46} & {1} & {SW2} & {SW\_SPDT} \\
\hline
{47} & {1} & {SW3} & {SW\_SPDT} \\
\hline
{48} & {1} & {U1} & {TPA711D} \\
\hline
{49} & {1} & {U2} & {SN76489AN} \\
\hline
{50} & {1} & {U3} & {Z84C0010AEG} \\
\hline
{51} & {1} & {U4} & {CY62256-55PC} \\
\hline
{52} & {1} & {U5} & {27C256} \\
\hline
{53} & {1} & {U6} & {TMS9118NL} \\
\hline
{54} & {2} & {U7, U8} & {TMS4416} \\
\hline
{55} & {1} & {U9} & {EPM7128SLC} \\
\hline
{56} & {1} & {U10} & {74ABT125} \\
\hline
{57} & {1} & {U11} & {YMZ284} \\
\hline
{58} & {5} & {U12, U13, U14, U15, U16} & {74AHCT1G08} \\
\hline
{59} & {1} & {Y1} & {10.738635 MHz} \\
\hline
\end{longtable}
\end{footnotesize}

\section{Building}

\par
This document assumes some Electrical Engineering knowledge. Building circuits is not
trivial due to the mix of SMD and through hole components. What follow are general
steps to build the Mini Colecovision

\begin{itemize}
  \item Create main PCB from schematic/gerber/coleco\_original.zip
  \item Create Right Angle PCB from schematic/gerber/right\_angle/right\_angle.zip
  \item Program ROM with BIOS
  \item Populate main PCB
  \item Populate right angle PCB
  \item Power up and program CPLD
  \item Build your own case
\end{itemize}

\subsection{Dependencies}

\par
The following are the dependencies needed to build the firmware and PCB for the system.

\begin{itemize}
  \item Quartus 13.0 sp1
  \item python 3.X
  \item KiCAD v7.X
\end{itemize}

\subsubsection{Protable Coleco Glue File List}
\begin{itemize}
\item src
	\begin{itemize}
	\item {'src/porta\_glue\_coleco.v': {'file\_type': 'verilogSource'}}
	\end{itemize}
\item constr
	\begin{itemize}
	\item {'constr/porta\_glue\_coleco.sdc': {'file\_type': 'SDC'}}
	\end{itemize}
\item tb
	\begin{itemize}
	\item {'tb/tb\_porta\_glue\_coleco.v': {'file\_type': 'verilogSource'}}
	\end{itemize}
\end{itemize}


\subsubsection{Fusesoc}
\subsubsection{Protable Coleco Glue Targets}
\begin{itemize}
\item default
	\begin{itemize}
	\item[$\space$] Info: Default IP target for future tool intergration.
	\item src
	\item constr
	\end{itemize}
\item sim
	\begin{itemize}
	\item[$\space$] Info: Simulation target for basic test bench.
	\item src
	\item tb
	\end{itemize}
\end{itemize}


\par
Fusesoc is used for the simulation target only. There are no build targets due to the use of Quartus 13.0sp1.
This makes the use of it a bit silly. It does make it easier to use in future projects where the RAM,ROM,CPU,VDP,
and Sound chips are also IP cores.

\subsubsection{Quartus}
\par
This project uses the last version of Quartus that supports the MAX7000S series. The version is 13.0sp1.
The project is located at src/quartus13sp01/. Once you have the project open please follow the softwares steps
for building and programming the CPLD bitfile.

\subsection{PCB}

\par
The four layer PCB is fairly easy to populate. The right angle PCB is a dual layer PCB which is even easier.
I recommend starting with resistors, then IC's, and then the rest. Surface mount parts should be done last.
This is a fairly complex project to build, take great caution in making sure your CPU and CPLD are installed
correctly. Its easy to rotate square packages these come in.

\subsection{3D Printed Case}

\par
A 3D printed case model is not included. I've kept this for release in the future.

\subsection{Programming}

\par
There are two devices that need to be programmed. ROM (read only memory) and the CPLD (complex programmable logic device).
They use two different methods to be programmed. The ROM is done off the board and then installed. The CPLD is installed
and the JTAG header is used to upload the bitfile.

\subsubsection{ROM}

\par
A TL866 is an excellent device for programming the ROM with a BIOS. The open source minipro application works well with it and
its clones. Below is a example command to use to program the ROM with a bios.

  \begin{lstlisting}[language=bash]
$ minipro -p ST27C256 -w coleco_bios.bin
  \end{lstlisting}

\subsubsection{CPLD}

\par
Quartus 13.0sp1 is the easiest way to build and program the MAX7000 CPLD. You will need an altera blaster.
I recommend the chinese clone blasters, they actually worked the best. While the worst was the Terasic blaster
which did not work at all. As for instructions on how to program it in Quartus, please see the software for details.

\newpage

\section{Usage}

\subsection{Directory Guide}

\par
Below highlights important folders from the root of mini\_colecovision.

\begin{enumerate}
  \item \textbf{docs} Contains all documentation related to this project.
    \begin{itemize}
      \item \textbf{datasheets} Contains all datasheets for components.
      \item \textbf{manual} Contains user manual and github page that are generated from the latex sources.
    \end{itemize}
  \item \textbf{img} Contains images of the project
  \item \textbf{schematic} KiCAD v7.X schematic and PCB designs
    \begin{itemize}
      \item \textbf{gerber} Contains gerber files and archives for production.
      \item \textbf{pdf} PDF schematic
    \end{itemize}
  \item \textbf{src} CPLD firmware source
    \begin{itemize}
      \item \textbf{protable\_coleco} Contains verilog source code and constraits
      \item \textbf{quartus13sp01} Quartus project to use to generate firmware file.
    \end{itemize}
\end{enumerate}
