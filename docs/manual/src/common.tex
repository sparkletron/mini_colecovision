\begin{titlepage}
  \begin{center}

  {\Huge Mini Colecovision}

  \vspace{25mm}

  \includegraphics[width=0.90\textwidth,height=\textheight,keepaspectratio]{img/SPARKLETRON.png}

  \vspace{25mm}

  \today

  \vspace{15mm}

  {\Large Jay Convertino}

  \end{center}
\end{titlepage}

\tableofcontents

\newpage

\section{Usage}

\subsection{Introduction}

\par
This manual describes how to use the RODAC (Retro Only Device Application Creation) system for development.
Items such as how to build included apps, what the structure of the system looks like, and how to create your
own app is included. The final section are links to doxygen generatated documentation about the drivers used
by this system.

\subsection{Dependecies}

\par
The following are the dependecies needed to build the applications targeting various retro systems.

\begin{itemize}
  \item sdcc 4.X.X
  \item python 3.X
  \item make
\end{itemize}

\subsubsection{Protable Coleco Glue File List}
\begin{itemize}
\item src
	\begin{itemize}
	\item {'src/porta\_glue\_coleco.v': {'file\_type': 'verilogSource'}}
	\end{itemize}
\item constr
	\begin{itemize}
	\item {'constr/porta\_glue\_coleco.sdc': {'file\_type': 'SDC'}}
	\end{itemize}
\item tb
	\begin{itemize}
	\item {'tb/tb\_porta\_glue\_coleco.v': {'file\_type': 'verilogSource'}}
	\end{itemize}
\end{itemize}

\subsubsection{Protable Coleco Glue Targets}
\begin{itemize}
\item default
	\begin{itemize}
	\item[$\space$] Info: Default IP target for future tool intergration.
	\item src
	\item constr
	\end{itemize}
\item sim
	\begin{itemize}
	\item[$\space$] Info: Simulation target for basic test bench.
	\item src
	\item tb
	\end{itemize}
\end{itemize}


\subsection{Building}

\par
Makefiles are used to execute all builds. All sources will rebuild when make is run due to the ability to change the
system target. If this didn't happen the new memory map setup in the defines.h would not be applied. Each application
has its makefile located in its root folder. To run a build you must run, in the target apps root folder, the following
command.

\begin{lstlisting}[language=bash]
  $ make SYSTEM
\end{lstlisting}

Where system is the target you would like to build for. All will do nothing but through an error telling you the same.
Currently the targets are \textbf{coleco, coleco\_sgm, msx, sg1000}. All targets have been tested for coleco based systems.
The others are not tested on real hardware at the moment.

\subsubsection{hello\_world}

\par
Hello World is a simple application that prints all of the characters from the TMS memory to screen. It also prints hello world in the
center of the screen and scrolls it. This is done in the TMS txt mode with 40 columns and no sprites. This application has been
tested in emulation on all available systems. It also generates a single annoying constant tone, as a really poor sound test. To build this
run the following for the Colecovision in the root of the apps/hello\_world folder.

\begin{lstlisting}[language=bash]
  $ make coleco
\end{lstlisting}

\subsubsection{multicart}

\par
Multicart creates a non-scrolling list of ROMs in alphabetical order. The number of ROMs is limited by the target flash size and the number of
lines on screen (till scrolling is added, currently 21). The generation of the header that contains the list of ROMs is automatic with a python
script, rom\_header\_gen.py. The full ROM is also auto generated by a python script rom\_file\_gen.py. Currently these default to the roms folder
located in the apps/multicart folder root. This can be changed in the make file via the ROM\_\* variables. There are dummy ROMs with random data
for testing of the system. This will work in an emulator up to the point of bank switching ROMs, since the PIC and the logic with it is not emulated.
This is currently only targeted and tested on the Colecovision. To build for the Colecovision you would run the following in the apps/multicart folder.

\begin{lstlisting}[language=bash]
  $ make coleco
\end{lstlisting}

\subsection{Directory Guide}

\par
Below highlights important folders from the root of RODAC.

\begin{enumerate}
  \item \textbf{docs} Contains all documentation related to this project.
    \begin{itemize}
      \item \textbf{arch} Contains all architecture docs related to retro systems.
      \item \textbf{manual} Contains user manual and wiki that are generated from the same latex source.
    \end{itemize}
  \item \textbf{apps} Contains source code in C for the applications to run on the target architecture.
    \begin{itemize}
      \item \textbf{hello\_world} Example hello world application. Targets all architectures.
      \item \textbf{mutlicart} Example multicart application, written for the coleco only.
    \end{itemize}
  \item \textbf{drivers} Contains all source code related to the project.
    \begin{itemize}
      \item \textbf{gisnd} Simple driver for the GI AY-3-8910 sound chip and its variants.
      \item \textbf{sn76489} driver for the TI SN76489 sound chip.
      \item \textbf{tms99XX} driver for all TMS99XX and TMS9XXX video chips.
    \end{itemize}
\end{enumerate}

\newpage

\section{Application Creation}

\newpage

\section{System Creation}

\section{Module Documentation}

\newpage
